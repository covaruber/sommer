\documentclass[]{article}
\usepackage{lmodern}
\usepackage{amssymb,amsmath}
\usepackage{ifxetex,ifluatex}
\usepackage{fixltx2e} % provides \textsubscript
\ifnum 0\ifxetex 1\fi\ifluatex 1\fi=0 % if pdftex
  \usepackage[T1]{fontenc}
  \usepackage[utf8]{inputenc}
\else % if luatex or xelatex
  \ifxetex
    \usepackage{mathspec}
  \else
    \usepackage{fontspec}
  \fi
  \defaultfontfeatures{Ligatures=TeX,Scale=MatchLowercase}
\fi
% use upquote if available, for straight quotes in verbatim environments
\IfFileExists{upquote.sty}{\usepackage{upquote}}{}
% use microtype if available
\IfFileExists{microtype.sty}{%
\usepackage{microtype}
\UseMicrotypeSet[protrusion]{basicmath} % disable protrusion for tt fonts
}{}
\usepackage[margin=1in]{geometry}
\usepackage{hyperref}
\hypersetup{unicode=true,
            pdftitle={Quantitative genetics using the sommer package},
            pdfauthor={Giovanny Covarrubias-Pazaran},
            pdfborder={0 0 0},
            breaklinks=true}
\urlstyle{same}  % don't use monospace font for urls
\usepackage{color}
\usepackage{fancyvrb}
\newcommand{\VerbBar}{|}
\newcommand{\VERB}{\Verb[commandchars=\\\{\}]}
\DefineVerbatimEnvironment{Highlighting}{Verbatim}{commandchars=\\\{\}}
% Add ',fontsize=\small' for more characters per line
\usepackage{framed}
\definecolor{shadecolor}{RGB}{248,248,248}
\newenvironment{Shaded}{\begin{snugshade}}{\end{snugshade}}
\newcommand{\KeywordTok}[1]{\textcolor[rgb]{0.13,0.29,0.53}{\textbf{#1}}}
\newcommand{\DataTypeTok}[1]{\textcolor[rgb]{0.13,0.29,0.53}{#1}}
\newcommand{\DecValTok}[1]{\textcolor[rgb]{0.00,0.00,0.81}{#1}}
\newcommand{\BaseNTok}[1]{\textcolor[rgb]{0.00,0.00,0.81}{#1}}
\newcommand{\FloatTok}[1]{\textcolor[rgb]{0.00,0.00,0.81}{#1}}
\newcommand{\ConstantTok}[1]{\textcolor[rgb]{0.00,0.00,0.00}{#1}}
\newcommand{\CharTok}[1]{\textcolor[rgb]{0.31,0.60,0.02}{#1}}
\newcommand{\SpecialCharTok}[1]{\textcolor[rgb]{0.00,0.00,0.00}{#1}}
\newcommand{\StringTok}[1]{\textcolor[rgb]{0.31,0.60,0.02}{#1}}
\newcommand{\VerbatimStringTok}[1]{\textcolor[rgb]{0.31,0.60,0.02}{#1}}
\newcommand{\SpecialStringTok}[1]{\textcolor[rgb]{0.31,0.60,0.02}{#1}}
\newcommand{\ImportTok}[1]{#1}
\newcommand{\CommentTok}[1]{\textcolor[rgb]{0.56,0.35,0.01}{\textit{#1}}}
\newcommand{\DocumentationTok}[1]{\textcolor[rgb]{0.56,0.35,0.01}{\textbf{\textit{#1}}}}
\newcommand{\AnnotationTok}[1]{\textcolor[rgb]{0.56,0.35,0.01}{\textbf{\textit{#1}}}}
\newcommand{\CommentVarTok}[1]{\textcolor[rgb]{0.56,0.35,0.01}{\textbf{\textit{#1}}}}
\newcommand{\OtherTok}[1]{\textcolor[rgb]{0.56,0.35,0.01}{#1}}
\newcommand{\FunctionTok}[1]{\textcolor[rgb]{0.00,0.00,0.00}{#1}}
\newcommand{\VariableTok}[1]{\textcolor[rgb]{0.00,0.00,0.00}{#1}}
\newcommand{\ControlFlowTok}[1]{\textcolor[rgb]{0.13,0.29,0.53}{\textbf{#1}}}
\newcommand{\OperatorTok}[1]{\textcolor[rgb]{0.81,0.36,0.00}{\textbf{#1}}}
\newcommand{\BuiltInTok}[1]{#1}
\newcommand{\ExtensionTok}[1]{#1}
\newcommand{\PreprocessorTok}[1]{\textcolor[rgb]{0.56,0.35,0.01}{\textit{#1}}}
\newcommand{\AttributeTok}[1]{\textcolor[rgb]{0.77,0.63,0.00}{#1}}
\newcommand{\RegionMarkerTok}[1]{#1}
\newcommand{\InformationTok}[1]{\textcolor[rgb]{0.56,0.35,0.01}{\textbf{\textit{#1}}}}
\newcommand{\WarningTok}[1]{\textcolor[rgb]{0.56,0.35,0.01}{\textbf{\textit{#1}}}}
\newcommand{\AlertTok}[1]{\textcolor[rgb]{0.94,0.16,0.16}{#1}}
\newcommand{\ErrorTok}[1]{\textcolor[rgb]{0.64,0.00,0.00}{\textbf{#1}}}
\newcommand{\NormalTok}[1]{#1}
\usepackage{graphicx,grffile}
\makeatletter
\def\maxwidth{\ifdim\Gin@nat@width>\linewidth\linewidth\else\Gin@nat@width\fi}
\def\maxheight{\ifdim\Gin@nat@height>\textheight\textheight\else\Gin@nat@height\fi}
\makeatother
% Scale images if necessary, so that they will not overflow the page
% margins by default, and it is still possible to overwrite the defaults
% using explicit options in \includegraphics[width, height, ...]{}
\setkeys{Gin}{width=\maxwidth,height=\maxheight,keepaspectratio}
\IfFileExists{parskip.sty}{%
\usepackage{parskip}
}{% else
\setlength{\parindent}{0pt}
\setlength{\parskip}{6pt plus 2pt minus 1pt}
}
\setlength{\emergencystretch}{3em}  % prevent overfull lines
\providecommand{\tightlist}{%
  \setlength{\itemsep}{0pt}\setlength{\parskip}{0pt}}
\setcounter{secnumdepth}{0}
% Redefines (sub)paragraphs to behave more like sections
\ifx\paragraph\undefined\else
\let\oldparagraph\paragraph
\renewcommand{\paragraph}[1]{\oldparagraph{#1}\mbox{}}
\fi
\ifx\subparagraph\undefined\else
\let\oldsubparagraph\subparagraph
\renewcommand{\subparagraph}[1]{\oldsubparagraph{#1}\mbox{}}
\fi

%%% Use protect on footnotes to avoid problems with footnotes in titles
\let\rmarkdownfootnote\footnote%
\def\footnote{\protect\rmarkdownfootnote}

%%% Change title format to be more compact
\usepackage{titling}

% Create subtitle command for use in maketitle
\newcommand{\subtitle}[1]{
  \posttitle{
    \begin{center}\large#1\end{center}
    }
}

\setlength{\droptitle}{-2em}

  \title{Quantitative genetics using the sommer package}
    \pretitle{\vspace{\droptitle}\centering\huge}
  \posttitle{\par}
    \author{Giovanny Covarrubias-Pazaran}
    \preauthor{\centering\large\emph}
  \postauthor{\par}
      \predate{\centering\large\emph}
  \postdate{\par}
    \date{2018-12-30}


\begin{document}
\maketitle

The sommer package was developed to provide R users a powerful and
reliable multivariate mixed model solver for different genetic and
non-genetic analysis in diploid and polyploid organisms. This package
allows the user to estimate variance components for a mixed model with
the advantage of specifying the variance-covariance structure of the
random effects, specify heterogeneous variances, and obtain other
parameters such as BLUPs, BLUEs, residuals, fitted values, variances for
fixed and random effects, etc. The core algorithms of the package are
coded in C++ using the Armadillo library to opmitime dense matrix
operations common in the derect-inversion algorithms.

The package is focused on problems of the type p \textgreater{} n
related to genomic prediction (hybrid prediction \& genomic selection)
and GWAS analysis, although any general mixed model can be fitted as
well. The package provides kernels to estimate additive
(\texttt{A.mat}), dominance (\texttt{D.mat}), and epistatic
(\texttt{E.mat}) relationship matrices that have been shown to increase
prediction accuracy under certain scenarios or simply to estimate the
variance components of such. The package provides flexibility to fit
other genetic models such as full and half diallel models as well.

Vignettes aim to provide several examples in how to use the sommer
package under different scenarios. We will spend the rest of the space
providing examples for:

\begin{enumerate}
\def\labelenumi{\arabic{enumi})}
\tightlist
\item
  Heritability (\(h^2\)) calculation
\item
  Specifying heterogeneous variances in mixed models
\item
  Using the pin calculator
\item
  Half and full diallel designs (using the overlay)
\item
  Genomic selection (predicting mendelian sampling)

  \begin{itemize}
  \tightlist
  \item
    GBLUP
  \item
    rrBLUP
  \end{itemize}
\item
  Single cross prediction (hybrid prediction)
\item
  Spatial modeling (using the 2-dimensional splines)
\item
  Multivariate genetic models and genetic correlations
\item
  Final remarks
\end{enumerate}

\subsection{Background}\label{background}

The core of the package the \texttt{mmer}function which solve the mixed
model equations. The functions are an interface to call the \texttt{NR}
Direct-Inversion Newton-Raphson or Average Information (Tunnicliffe
1989; Gilmour et al. 1995; Lee et al. 2016). Since version 2.0 sommer
can handle multivariate models. Following Maier et al. (2015), the
multivariate (and by extension the univariate) mixed model implemented
has the form:

\(y_1 = X_1\beta_1 + Z_1u_1 + \epsilon_1\)
\(y_2 = X_2\beta_2 + Z_2u_2 + \epsilon_2\) \ldots{}
\(y_i = X_i\beta_i + Z_iu_i + \epsilon_i\)

where \(y_i\) is a vector of trait phenotypes, \(\beta_i\) is a vector
of fixed effects, \(u_i\) is a vector of random effects for individuals
and \(e_i\) are residuals for trait `i' (i = 1, \ldots{}, t). The random
effects (\(u_1\) \ldots{} \(u_i\) and \(e_i\)) are assumed to be
normally distributed with mean zero. X and Z are incidence matrices for
fixed and random effects respectively. The distribution of the
multivariate response and the phenotypic variance covariance (V) are:

\(Y = X\beta + ZU + \epsilon_i\)

Y \textasciitilde{} MVN(\(X\beta\), V)

\[\mathbf{Y} = \left[\begin{array}
{r}
y_1 \\
y_2 \\
... \\
y_t \\
\end{array}\right]
\]

\[\mathbf{X} = \left[\begin{array}
{rrr}
X_1 & ... & ... \\
\vdots & \ddots & \vdots\\
... & ... & X_t \\
\end{array}\right]
\]

\[\mathbf{V} = \left[\begin{array}
{rrr}
Z_1 K{\sigma^2_{g_{1}}} Z_1' + H{\sigma^2_{\epsilon_{1}}} & ... & Z_1 K{\sigma_{g_{1,t}}} Z_t' + H{\sigma_{\epsilon_{1,t}}}\\
 \vdots & \ddots & \vdots\\
Z_1 K{\sigma_{g_{1,t}}} Z_t' + H{\sigma_{\epsilon_{1,t}}} & ... & Z_t K{\sigma^2_{g_{t}}} Z_t' + H{\sigma^2_{\epsilon_{t}}} \\
\end{array}\right]
\]

where K is the relationship or covariance matrix for the kth random
effect (u=1,\ldots{},k), and R=I is an identity matrix for the residual
term. The terms \(\sigma^2_{g_{i}}\) and \(\sigma^2_{\epsilon_{i}}\)
denote the genetic (or any of the kth random terms) and residual
variance of trait `i', respectively and \(\sigma_{g_{_{ij}}}\) and
\(\sigma_{\epsilon_{_{ij}}}\) the genetic (or any of the kth random
terms) and residual covariance between traits `i' and `j'
(i=1,\ldots{},t, and j=1,\ldots{},t). The algorithm implemented
optimizes the log likelihood:

\(logL = 1/2 * ln(|V|) + ln(X'|V|X) + Y'PY\)

where \textbar{}\textbar{} is the determinant of a matrix. And the REML
estimates are updated using a Newton optimization algorithm of the form:

\(\theta^{k+1} = \theta^{k} + (H^{k})^{-1}*\frac{dL}{d\sigma^2_i}|\theta^k\)

Where, \(\theta\) is the vector of variance components for random
effects and covariance components among traits, \(H^{-1}\) is the
inverse of the Hessian matrix of second derivatives for the kth cycle,
\(\frac{dL}{d\sigma^2_i}\) is the vector of first derivatives of the
likelihood with respect to the variance-covariance components. The Eigen
decomposition of the relationship matrix proposed by Lee and Van Der
Werf (2016) was included in the Newton-Raphson algorithm to improve time
efficiency. Additionally, the popular pin function to estimate standard
errors for linear combinations of variance components
(i.e.~heritabilities and genetic correlations) was added to the package
as well.

Please refer to the canonical papers listed in the Literature section to
check how the algorithms work. We have tested widely the methods to make
sure they provide the same solution when the likelihood behaves well but
for complex problems they might lead to slightly different answers. If
you have any concern please contact me at
\href{mailto:cova_ruber@live.com.mx}{\nolinkurl{cova\_ruber@live.com.mx}}.

In the following section we will go in detail over several examples on
how to use mixed models in univariate and multivariate case and their
use in quantitative genetics.

\subsection{1) Marker and non-marker based heritability
calculation}\label{marker-and-non-marker-based-heritability-calculation}

The heritability is one of the most popular parameters among the
breeding and genetics community because of the insight that provides in
the inheritance of the trait. The heritability is usually estimated as
narrow sense (\(h^2\); only additive variance in the numerator
\(\sigma^2_A\)), and broad sense (\(H^2\); all genetic variance in the
numerator \(\sigma^2_G\)).

In a classical breeding experiment with no molecular markers, special
designs are performed to estimate and disect the additive
(\(\sigma^2_A\)) and non-additive (i.e.~dominance \(\sigma^2_D\))
variance along with environmental variability. Designs such as
generation analysis, North Carolina designs are used to disect
\(\sigma^2_A\) and \(\sigma^2_D\) to estimate the narrow sense
heritability (\(h^2\)). When no special design is available we can still
disect the genetic variance (\(\sigma^2_G\)) and estimate the broad
sense heritability. In this first example we will show the broad sense
estimation which doesn't use covariance structures for the genotipic
effect (i.e.~genomic or additive relationship matrices). For big models
with no covariance structures, sommer's direct inversion is a bad idea
to use but we will show anyways how to do it, but keep in mind that for
very sparse models we recommend using the lmer function from the lme4
package or any other package using MME-based algorithms (i.e.~asreml-R).

The following dataset has 41 potato lines evaluated in 5 locations
across 3 years in an RCBD design. We show how to fit the model and
extract the variance components to calculate the \(h^2\).

\begin{Shaded}
\begin{Highlighting}[]
\KeywordTok{library}\NormalTok{(sommer)}
\KeywordTok{data}\NormalTok{(DT_example)}
\KeywordTok{head}\NormalTok{(DT)}
\end{Highlighting}
\end{Shaded}

\begin{verbatim}
##                   Name     Env Loc Year     Block Yield    Weight
## 33  Manistee(MSL292-A) CA.2013  CA 2013 CA.2013.1     4 -1.904711
## 65          CO02024-9W CA.2013  CA 2013 CA.2013.1     5 -1.446958
## 66  Manistee(MSL292-A) CA.2013  CA 2013 CA.2013.2     5 -1.516271
## 67            MSL007-B CA.2011  CA 2011 CA.2011.2     5 -1.435510
## 68           MSR169-8Y CA.2013  CA 2013 CA.2013.1     5 -1.469051
## 103         AC05153-1W CA.2013  CA 2013 CA.2013.1     6 -1.307167
\end{verbatim}

\begin{Shaded}
\begin{Highlighting}[]
\NormalTok{ans1 <-}\StringTok{ }\KeywordTok{mmer}\NormalTok{(Yield}\OperatorTok{~}\DecValTok{1}\NormalTok{,}
             \DataTypeTok{random=} \OperatorTok{~}\StringTok{ }\NormalTok{Name }\OperatorTok{+}\StringTok{ }\NormalTok{Env }\OperatorTok{+}\StringTok{ }\NormalTok{Env}\OperatorTok{:}\NormalTok{Name }\OperatorTok{+}\StringTok{ }\NormalTok{Env}\OperatorTok{:}\NormalTok{Block,}
             \DataTypeTok{rcov=} \OperatorTok{~}\StringTok{ }\NormalTok{units,}
             \DataTypeTok{data=}\NormalTok{DT)}
\end{Highlighting}
\end{Shaded}

\begin{verbatim}
## iteration    LogLik     wall    cpu(sec)   restrained
##     1      -40.765   13:35:24      0           0
##     2      -30.2657   13:35:24      0           0
##     3      -25.8227   13:35:24      0           1
##     4      -24.7277   13:35:24      0           1
##     5      -24.7203   13:35:24      0           1
##     6      -24.7202   13:35:24      0           1
\end{verbatim}

\begin{Shaded}
\begin{Highlighting}[]
\KeywordTok{summary}\NormalTok{(ans1)}\OperatorTok{$}\NormalTok{varcomp}
\end{Highlighting}
\end{Shaded}

\begin{verbatim}
##                         VarComp  VarCompSE    Zratio Constraint
## Name.Yield-Yield       3.718279  1.6959834 2.1924029   Positive
## Env.Yield-Yield       12.008450 12.2771178 0.9781164   Positive
## Env:Name.Yield-Yield   5.152643  1.4923912 3.4526091   Positive
## Env:Block.Yield-Yield  0.000000  0.1156675 0.0000000   Positive
## units.Yield-Yield      4.366189  0.6573086 6.6425245   Positive
\end{verbatim}

\begin{Shaded}
\begin{Highlighting}[]
\NormalTok{(n.env <-}\StringTok{ }\KeywordTok{length}\NormalTok{(}\KeywordTok{levels}\NormalTok{(DT}\OperatorTok{$}\NormalTok{Env)))}
\end{Highlighting}
\end{Shaded}

\begin{verbatim}
## [1] 3
\end{verbatim}

\begin{Shaded}
\begin{Highlighting}[]
\KeywordTok{pin}\NormalTok{(ans1, h2 }\OperatorTok{~}\StringTok{ }\NormalTok{V1 }\OperatorTok{/}\StringTok{ }\NormalTok{( V1 }\OperatorTok{+}\StringTok{ }\NormalTok{(V3}\OperatorTok{/}\NormalTok{n.env) }\OperatorTok{+}\StringTok{ }\NormalTok{(V5}\OperatorTok{/}\NormalTok{(}\DecValTok{2}\OperatorTok{*}\NormalTok{n.env)) ) )}
\end{Highlighting}
\end{Shaded}

\begin{verbatim}
##     Estimate        SE
## h2 0.6032715 0.1344582
\end{verbatim}

Recently with markers becoming cheaper, thousand of markers can be run
in the breeding materials. When markers are available, an special design
is not neccesary to disect the additive genetic variance. The
availability of the additive, dominance and epistatic relationship
matrices allow us to estimate \(\sigma^2_A\), \(\sigma^2_D\) and
\(\sigma^2_I\), although given that A, D and E are not orthogonal the
interpretation of models that fit more than A and D become cumbersome.

Assume you have a population (even unreplicated) in the field but in
addition we have genetic markers. Now we can fit the model and estimate
the genomic heritability that explains a portion of the additive genetic
variance (with high marker density \(\sigma^2_A\) = \(\sigma^2_g\))

\begin{Shaded}
\begin{Highlighting}[]
\KeywordTok{data}\NormalTok{(}\StringTok{"DT_cpdata"}\NormalTok{)}
\NormalTok{DT}\OperatorTok{$}\NormalTok{idd <-DT}\OperatorTok{$}\NormalTok{id; DT}\OperatorTok{$}\NormalTok{ide <-DT}\OperatorTok{$}\NormalTok{id}
\NormalTok{### look at the data}
\NormalTok{A <-}\StringTok{ }\KeywordTok{A.mat}\NormalTok{(GT) }\CommentTok{# additive relationship matrix}
\NormalTok{D <-}\StringTok{ }\KeywordTok{D.mat}\NormalTok{(GT) }\CommentTok{# dominance relationship matrix}
\NormalTok{E <-}\StringTok{ }\KeywordTok{E.mat}\NormalTok{(GT) }\CommentTok{# epistatic relationship matrix}
\NormalTok{ans.ADE <-}\StringTok{ }\KeywordTok{mmer}\NormalTok{(color}\OperatorTok{~}\DecValTok{1}\NormalTok{, }
                 \DataTypeTok{random=}\OperatorTok{~}\KeywordTok{vs}\NormalTok{(id,}\DataTypeTok{Gu=}\NormalTok{A) }\OperatorTok{+}\StringTok{ }\KeywordTok{vs}\NormalTok{(idd,}\DataTypeTok{Gu=}\NormalTok{D), }
                 \DataTypeTok{rcov=}\OperatorTok{~}\NormalTok{units,}
                 \DataTypeTok{data=}\NormalTok{DT)}
\end{Highlighting}
\end{Shaded}

\begin{verbatim}
## iteration    LogLik     wall    cpu(sec)   restrained
##     1      -123   13:35:26      0           0
##     2      -107.864   13:35:27      1           0
##     3      -103.867   13:35:27      1           0
##     4      -103.315   13:35:27      1           0
##     5      -103.294   13:35:28      2           0
##     6      -103.293   13:35:28      2           0
\end{verbatim}

\begin{Shaded}
\begin{Highlighting}[]
\NormalTok{(}\KeywordTok{summary}\NormalTok{(ans.ADE)}\OperatorTok{$}\NormalTok{varcomp)}
\end{Highlighting}
\end{Shaded}

\begin{verbatim}
##                       VarComp    VarCompSE   Zratio Constraint
## u:id.color-color  0.003662202 0.0012194130 3.003250   Positive
## u:idd.color-color 0.001820079 0.0007406216 2.457502   Positive
## units.color-color 0.002106929 0.0002864724 7.354736   Positive
\end{verbatim}

\begin{Shaded}
\begin{Highlighting}[]
\KeywordTok{pin}\NormalTok{(ans.ADE, h2 }\OperatorTok{~}\StringTok{ }\NormalTok{(V1) }\OperatorTok{/}\StringTok{ }\NormalTok{( V1}\OperatorTok{+}\NormalTok{V3) )}
\end{Highlighting}
\end{Shaded}

\begin{verbatim}
##     Estimate         SE
## h2 0.6347926 0.08840488
\end{verbatim}

\begin{Shaded}
\begin{Highlighting}[]
\KeywordTok{pin}\NormalTok{(ans.ADE, h2 }\OperatorTok{~}\StringTok{ }\NormalTok{(V1}\OperatorTok{+}\NormalTok{V2) }\OperatorTok{/}\StringTok{ }\NormalTok{( V1}\OperatorTok{+}\NormalTok{V2}\OperatorTok{+}\NormalTok{V3) )}
\end{Highlighting}
\end{Shaded}

\begin{verbatim}
##     Estimate         SE
## h2 0.7223783 0.05563774
\end{verbatim}

In the previous example we showed how to estimate the additive
(\(\sigma^2_A\)), dominance (\(\sigma^2_D\)), and epistatic
(\(\sigma^2_I\)) variance components based on markers and estimate broad
(\(H^2\)) and narrow sense heritability (\(h^2\)). Notice that we used
the \texttt{vs()} function which indicates that the random effect inside
the parenthesis (i.e.~id, idd or ide) has a covariance matrix (A, D, or
E), that will be specified in the Gu argument of the vs() function.
Please DO NOT provide the inverse but the original covariance matrix.

\subsection{2) Specifying heterogeneous variances in univariate
models}\label{specifying-heterogeneous-variances-in-univariate-models}

Very often in multi-environment trials, the assumption that genetic
variance is the same across locations may be too naive. Because of that,
specifying a general genetic component and a location specific genetic
variance is the way to go.

We estimate variance components for \(GCA_2\) and \(SCA\) specifying the
variance structure.

\begin{Shaded}
\begin{Highlighting}[]
\KeywordTok{data}\NormalTok{(}\StringTok{"DT_cornhybrids"}\NormalTok{)}
\NormalTok{### fit the model}
\NormalTok{modFD <-}\StringTok{ }\KeywordTok{mmer}\NormalTok{(Yield}\OperatorTok{~}\DecValTok{1}\NormalTok{, }
               \DataTypeTok{random=}\OperatorTok{~}\StringTok{ }\KeywordTok{vs}\NormalTok{(}\KeywordTok{at}\NormalTok{(Location,}\KeywordTok{c}\NormalTok{(}\StringTok{"3"}\NormalTok{,}\StringTok{"4"}\NormalTok{)),GCA2), }
               \DataTypeTok{rcov=} \OperatorTok{~}\StringTok{ }\KeywordTok{vs}\NormalTok{(}\KeywordTok{ds}\NormalTok{(Location),units),}
               \DataTypeTok{data=}\NormalTok{DT)}
\end{Highlighting}
\end{Shaded}

\begin{verbatim}
## iteration    LogLik     wall    cpu(sec)   restrained
##     1      -190.104   13:35:29      0           0
##     2      -171.543   13:35:30      1           0
##     3      -165.319   13:35:30      1           0
##     4      -164.691   13:35:31      2           0
##     5      -164.684   13:35:31      2           0
##     6      -164.684   13:35:32      3           0
\end{verbatim}

\begin{Shaded}
\begin{Highlighting}[]
\KeywordTok{summary}\NormalTok{(modFD)}
\end{Highlighting}
\end{Shaded}

\begin{verbatim}
## ============================================================
##          Multivariate Linear Mixed Model fit by REML         
## **********************  sommer 3.7  ********************** 
## ============================================================
##          logLik      AIC      BIC Method Converge
## Value -164.6839 331.3677 335.3592     NR     TRUE
## ============================================================
## Variance-Covariance components:
##                     VarComp VarCompSE Zratio Constraint
## 3:GCA2.Yield-Yield    62.48     53.45  1.169   Positive
## 4:GCA2.Yield-Yield    97.99     79.56  1.232   Positive
## 1:units.Yield-Yield  216.82     30.77  7.047   Positive
## 2:units.Yield-Yield  216.82     30.77  7.047   Positive
## 3:units.Yield-Yield  493.05     77.27  6.381   Positive
## 4:units.Yield-Yield  711.98    111.63  6.378   Positive
## ============================================================
## Fixed effects:
##   Trait      Effect Estimate Std.Error t.value
## 1 Yield (Intercept)    138.1    0.9442   146.3
## ============================================================
## Groups and observations:
##        Yield
## 3:GCA2    20
## 4:GCA2    20
## ============================================================
## Use the '$' sign to access results and parameters
\end{verbatim}

In the previous example we showed how the \texttt{at()} function is used
in the \texttt{mmer} solver. By using the \texttt{at} function you can
specify that i.e.~the GCA2 has a different variance in different
Locations, in this case locations 3 and 4, but also a main GCA variance.
This is considered a CS + DIAG (compound symmetry + diagonal) model.

In addition, other functions can be added on top to fit models with
covariance structures, i.e.~the Gu argument from the \texttt{vs()}
function to indicate a covariance matrix (A, pedigree or genomic
relationship matrix)

\begin{Shaded}
\begin{Highlighting}[]
\KeywordTok{data}\NormalTok{(}\StringTok{"DT_cornhybrids"}\NormalTok{)}
\NormalTok{GT[}\DecValTok{1}\OperatorTok{:}\DecValTok{4}\NormalTok{,}\DecValTok{1}\OperatorTok{:}\DecValTok{4}\NormalTok{]}
\end{Highlighting}
\end{Shaded}

\begin{verbatim}
##             A258       A634        A641        A680
## A258  2.23285528 -0.3504778 -0.04756856 -0.32239362
## A634 -0.35047780  1.4529169  0.45203869 -0.02293680
## A641 -0.04756856  0.4520387  1.96940221 -0.09896791
## A680 -0.32239362 -0.0229368 -0.09896791  1.65221984
\end{verbatim}

\begin{Shaded}
\begin{Highlighting}[]
\NormalTok{### fit the model}
\NormalTok{modFD <-}\StringTok{ }\KeywordTok{mmer}\NormalTok{(Yield}\OperatorTok{~}\DecValTok{1}\NormalTok{, }
              \DataTypeTok{random=}\OperatorTok{~}\StringTok{ }\KeywordTok{vs}\NormalTok{(}\KeywordTok{at}\NormalTok{(Location,}\KeywordTok{c}\NormalTok{(}\StringTok{"3"}\NormalTok{,}\StringTok{"4"}\NormalTok{)),GCA2,}\DataTypeTok{Gu=}\NormalTok{GT), }
              \DataTypeTok{rcov=} \OperatorTok{~}\StringTok{ }\KeywordTok{vs}\NormalTok{(}\KeywordTok{ds}\NormalTok{(Location),units),}
              \DataTypeTok{data=}\NormalTok{DT)}
\end{Highlighting}
\end{Shaded}

\begin{verbatim}
## iteration    LogLik     wall    cpu(sec)   restrained
##     1      -191.286   13:35:32      0           0
##     2      -172.247   13:35:33      1           0
##     3      -165.948   13:35:33      1           0
##     4      -165.248   13:35:34      2           0
##     5      -165.23   13:35:34      2           0
##     6      -165.229   13:35:35      3           0
##     7      -165.229   13:35:35      3           0
\end{verbatim}

\begin{Shaded}
\begin{Highlighting}[]
\KeywordTok{summary}\NormalTok{(modFD)}
\end{Highlighting}
\end{Shaded}

\begin{verbatim}
## ============================================================
##          Multivariate Linear Mixed Model fit by REML         
## **********************  sommer 3.7  ********************** 
## ============================================================
##          logLik      AIC      BIC Method Converge
## Value -165.2286 332.4571 336.4486     NR     TRUE
## ============================================================
## Variance-Covariance components:
##                     VarComp VarCompSE Zratio Constraint
## 3:GCA2.Yield-Yield    26.64     26.16 1.0185   Positive
## 4:GCA2.Yield-Yield    37.51     37.78 0.9927   Positive
## 1:units.Yield-Yield  216.77     30.75 7.0489   Positive
## 2:units.Yield-Yield  216.77     30.75 7.0489   Positive
## 3:units.Yield-Yield  503.62     77.87 6.4673   Positive
## 4:units.Yield-Yield  738.86    114.17 6.4715   Positive
## ============================================================
## Fixed effects:
##   Trait      Effect Estimate Std.Error t.value
## 1 Yield (Intercept)    138.1    0.9147     151
## ============================================================
## Groups and observations:
##        Yield
## 3:GCA2    20
## 4:GCA2    20
## ============================================================
## Use the '$' sign to access results and parameters
\end{verbatim}

\subsection{3) Using the pin calculator}\label{using-the-pin-calculator}

Sometimes the user needs to calculate ratios or functions of specific
variance-covariance components and obtain the standard error for such
parameters. Examples of these are the genetic correlations,
heritabilities, etc. Using the CPdata we will show how to estimate the
heritability and the standard error using the pin function that uses the
delta method to come up with these parameters. This can be extended for
any linear combination of the variance components.

\begin{Shaded}
\begin{Highlighting}[]
\KeywordTok{data}\NormalTok{(}\StringTok{"DT_cpdata"}\NormalTok{)}
\NormalTok{### look at the data}
\NormalTok{A <-}\StringTok{ }\KeywordTok{A.mat}\NormalTok{(GT) }\CommentTok{# additive relationship matrix}
\NormalTok{ans <-}\StringTok{ }\KeywordTok{mmer}\NormalTok{(color}\OperatorTok{~}\DecValTok{1}\NormalTok{, }
                \DataTypeTok{random=}\OperatorTok{~}\KeywordTok{vs}\NormalTok{(id,}\DataTypeTok{Gu=}\NormalTok{A), }
                \DataTypeTok{rcov=}\OperatorTok{~}\NormalTok{units,}
                \DataTypeTok{data=}\NormalTok{DT)}
\end{Highlighting}
\end{Shaded}

\begin{verbatim}
## iteration    LogLik     wall    cpu(sec)   restrained
##     1      -137.304   13:35:36      0           0
##     2      -115.507   13:35:36      0           0
##     3      -111.236   13:35:36      0           0
##     4      -110.755   13:35:37      1           0
##     5      -110.741   13:35:37      1           0
##     6      -110.741   13:35:37      1           0
\end{verbatim}

\begin{Shaded}
\begin{Highlighting}[]
\NormalTok{(}\KeywordTok{summary}\NormalTok{(ans.ADE)}\OperatorTok{$}\NormalTok{varcomp)}
\end{Highlighting}
\end{Shaded}

\begin{verbatim}
##                       VarComp    VarCompSE   Zratio Constraint
## u:id.color-color  0.003662202 0.0012194130 3.003250   Positive
## u:idd.color-color 0.001820079 0.0007406216 2.457502   Positive
## units.color-color 0.002106929 0.0002864724 7.354736   Positive
\end{verbatim}

\begin{Shaded}
\begin{Highlighting}[]
\KeywordTok{pin}\NormalTok{(ans, h2 }\OperatorTok{~}\StringTok{ }\NormalTok{(V1) }\OperatorTok{/}\StringTok{ }\NormalTok{( V1}\OperatorTok{+}\NormalTok{V2) )}
\end{Highlighting}
\end{Shaded}

\begin{verbatim}
##     Estimate         SE
## h2 0.6512157 0.06107574
\end{verbatim}

The same can be used for multivariate models. Please check the
documentation of the \texttt{pin} function to see more examples.

\subsection{4) Half and full diallel designs (use of the
overlay)}\label{half-and-full-diallel-designs-use-of-the-overlay}

When breeders are looking for the best single cross combinations,
diallel designs have been by far the most used design in crops like
maize. There are 4 types of diallel designs depending if reciprocate and
self cross (omission of parents) are performed (full diallel with
parents n\^{}2; full diallel without parents n(n-1); half diallel with
parents 1/2 * n(n+1); half diallel without parents 1/2 * n(n-1) ). In
this example we will show a full dialle design (reciprocate crosses are
performed) and half diallel designs (only one of the directions is
performed).

In the first data set we show a full diallel among 40 lines from 2
heterotic groups, 20 in each. Therefore 400 possible hybrids are
possible. We have pehnotypic data for 100 of them across 4 locations. We
use the data available to fit a model of the form:

\(y = X\beta + Zu_1 + Zu_2 + Zu_S + \epsilon\)

We estimate variance components for \(GCA_1\), \(GCA_2\) and \(SCA\) and
use them to estimate heritability. Additionally BLUPs for GCA and SCA
effects can be used to predict crosses.

\begin{Shaded}
\begin{Highlighting}[]
\KeywordTok{data}\NormalTok{(}\StringTok{"DT_cornhybrids"}\NormalTok{)}

\NormalTok{modFD <-}\StringTok{ }\KeywordTok{mmer}\NormalTok{(Yield}\OperatorTok{~}\NormalTok{Location, }
               \DataTypeTok{random=}\OperatorTok{~}\NormalTok{GCA1}\OperatorTok{+}\NormalTok{GCA2}\OperatorTok{+}\NormalTok{SCA, }
               \DataTypeTok{rcov=}\OperatorTok{~}\NormalTok{units,}
               \DataTypeTok{data=}\NormalTok{DT)}
\end{Highlighting}
\end{Shaded}

\begin{verbatim}
## iteration    LogLik     wall    cpu(sec)   restrained
##     1      -149.436   13:35:38      0           0
##     2      -136.475   13:35:38      0           1
##     3      -132.852   13:35:39      1           1
##     4      -132.625   13:35:39      1           1
##     5      -132.596   13:35:39      1           1
##     6      -132.59   13:35:40      2           1
##     7      -132.589   13:35:40      2           1
##     8      -132.589   13:35:41      3           1
\end{verbatim}

\begin{Shaded}
\begin{Highlighting}[]
\NormalTok{(suma <-}\StringTok{ }\KeywordTok{summary}\NormalTok{(modFD)}\OperatorTok{$}\NormalTok{varcomp)}
\end{Highlighting}
\end{Shaded}

\begin{verbatim}
##                      VarComp VarCompSE     Zratio Constraint
## GCA1.Yield-Yield    0.000000  16.50337  0.0000000   Positive
## GCA2.Yield-Yield    7.412226  18.94200  0.3913116   Positive
## SCA.Yield-Yield   187.560303  41.59428  4.5092817   Positive
## units.Yield-Yield 221.142463  18.14716 12.1860656   Positive
\end{verbatim}

\begin{Shaded}
\begin{Highlighting}[]
\NormalTok{Vgca <-}\StringTok{ }\KeywordTok{sum}\NormalTok{(suma[}\DecValTok{1}\OperatorTok{:}\DecValTok{2}\NormalTok{,}\DecValTok{1}\NormalTok{])}
\NormalTok{Vsca <-}\StringTok{ }\NormalTok{suma[}\DecValTok{3}\NormalTok{,}\DecValTok{1}\NormalTok{]}
\NormalTok{Ve <-}\StringTok{ }\NormalTok{suma[}\DecValTok{4}\NormalTok{,}\DecValTok{1}\NormalTok{]}
\NormalTok{Va =}\StringTok{ }\DecValTok{4}\OperatorTok{*}\NormalTok{Vgca}
\NormalTok{Vd =}\StringTok{ }\DecValTok{4}\OperatorTok{*}\NormalTok{Vsca}
\NormalTok{Vg <-}\StringTok{ }\NormalTok{Va }\OperatorTok{+}\StringTok{ }\NormalTok{Vd}
\NormalTok{(H2 <-}\StringTok{ }\NormalTok{Vg }\OperatorTok{/}\StringTok{ }\NormalTok{(Vg }\OperatorTok{+}\StringTok{ }\NormalTok{(Ve)) )}
\end{Highlighting}
\end{Shaded}

\begin{verbatim}
## [1] 0.7790856
\end{verbatim}

\begin{Shaded}
\begin{Highlighting}[]
\NormalTok{(h2 <-}\StringTok{ }\NormalTok{Va }\OperatorTok{/}\StringTok{ }\NormalTok{(Vg }\OperatorTok{+}\StringTok{ }\NormalTok{(Ve)) )}
\end{Highlighting}
\end{Shaded}

\begin{verbatim}
## [1] 0.02961832
\end{verbatim}

Don't worry too much about the small h2 value, the data was simulated to
be mainly dominance variance, therefore the Va was simulated extremely
small leading to such value of narrow sense h2.

In this second data set we show a small half diallel with 7 parents
crossed in one direction. n(n-1)/2 crosses are possible 7(6)/2 = 21
unique crosses. Parents appear as males or females indistictly. Each
with two replications in a CRD. For a half diallel design a single GCA
variance component for both males and females can be estimated and an
SCA as well (\(\sigma^2_GCA\) and \(\sigma^2_SCA\) respectively), and
BLUPs for GCA and SCA of the parents can be extracted. We would show
first how to use it with the \texttt{mmer} function using the
\texttt{overlay()} function. The specific model here is:

\(y = X\beta + Zu_g + Zu_s + \epsilon\)

\begin{Shaded}
\begin{Highlighting}[]
\KeywordTok{data}\NormalTok{(}\StringTok{"DT_halfdiallel"}\NormalTok{)}
\KeywordTok{head}\NormalTok{(DT)}
\end{Highlighting}
\end{Shaded}

\begin{verbatim}
##   rep geno male female     sugar
## 1   1   12    1      2 13.950509
## 2   2   12    1      2  9.756918
## 3   1   13    1      3 13.906355
## 4   2   13    1      3  9.119455
## 5   1   14    1      4  5.174483
## 6   2   14    1      4  8.452221
\end{verbatim}

\begin{Shaded}
\begin{Highlighting}[]
\NormalTok{DT}\OperatorTok{$}\NormalTok{femalef <-}\StringTok{ }\KeywordTok{as.factor}\NormalTok{(DT}\OperatorTok{$}\NormalTok{female)}
\NormalTok{DT}\OperatorTok{$}\NormalTok{malef <-}\StringTok{ }\KeywordTok{as.factor}\NormalTok{(DT}\OperatorTok{$}\NormalTok{male)}
\NormalTok{DT}\OperatorTok{$}\NormalTok{genof <-}\StringTok{ }\KeywordTok{as.factor}\NormalTok{(DT}\OperatorTok{$}\NormalTok{geno)}
\NormalTok{#### model using overlay}
\NormalTok{modh <-}\StringTok{ }\KeywordTok{mmer}\NormalTok{(sugar}\OperatorTok{~}\DecValTok{1}\NormalTok{, }
             \DataTypeTok{random=}\OperatorTok{~}\KeywordTok{vs}\NormalTok{(}\KeywordTok{overlay}\NormalTok{(femalef,malef)) }
             \OperatorTok{+}\StringTok{ }\NormalTok{genof,}
             \DataTypeTok{data=}\NormalTok{DT)}
\end{Highlighting}
\end{Shaded}

\begin{verbatim}
## iteration    LogLik     wall    cpu(sec)   restrained
##     1      -10.425   13:35:41      0           0
##     2      -6.487   13:35:41      0           0
##     3      -5.732   13:35:41      0           0
##     4      -5.67494   13:35:41      0           0
##     5      -5.67441   13:35:41      0           0
\end{verbatim}

\begin{Shaded}
\begin{Highlighting}[]
\KeywordTok{summary}\NormalTok{(modh)}\OperatorTok{$}\NormalTok{varcomp}
\end{Highlighting}
\end{Shaded}

\begin{verbatim}
##                        VarComp VarCompSE   Zratio Constraint
## u:femalef.sugar-sugar 5.507899 3.5741151 1.541052   Positive
## genof.sugar-sugar     1.815784 1.3629575 1.332238   Positive
## units.sugar-sugar     3.117538 0.9626094 3.238632   Positive
\end{verbatim}

Notice how the \texttt{overlay()} argument makes the overlap of
incidence matrices possible making sure that male and female are joint
into a single random effect.

\subsection{5) Genomic selection}\label{genomic-selection}

In this section we will use wheat data from CIMMYT to show how is
genomic selection performed. This is the case of prediction of specific
individuals within a population. It basically uses a similar model of
the form:

\(y = X\beta + Zu + \epsilon\)

and takes advantage of the variance covariance matrix for the genotype
effect known as the additive relationship matrix (A) and calculated
using the \texttt{A.mat} function to establish connections among all
individuals and predict the BLUPs for individuals that were not
measured. The prediction accuracy depends on several factors such as the
heritability (\(h^2\)), training population used (TP), size of TP, etc.

\begin{Shaded}
\begin{Highlighting}[]
\KeywordTok{data}\NormalTok{(}\StringTok{"DT_wheat"}\NormalTok{); }
\KeywordTok{colnames}\NormalTok{(DT) <-}\StringTok{ }\KeywordTok{paste0}\NormalTok{(}\StringTok{"X"}\NormalTok{,}\DecValTok{1}\OperatorTok{:}\KeywordTok{ncol}\NormalTok{(DT))}
\NormalTok{DT <-}\StringTok{ }\KeywordTok{as.data.frame}\NormalTok{(DT);DT}\OperatorTok{$}\NormalTok{id <-}\StringTok{ }\KeywordTok{as.factor}\NormalTok{(}\KeywordTok{rownames}\NormalTok{(DT))}
\CommentTok{# select environment 1}
\KeywordTok{rownames}\NormalTok{(GT) <-}\StringTok{ }\KeywordTok{rownames}\NormalTok{(DT)}
\NormalTok{K <-}\StringTok{ }\KeywordTok{A.mat}\NormalTok{(GT) }\CommentTok{# additive relationship matrix}
\KeywordTok{colnames}\NormalTok{(K) <-}\StringTok{ }\KeywordTok{rownames}\NormalTok{(K) <-}\StringTok{ }\KeywordTok{rownames}\NormalTok{(DT)}
\CommentTok{# GBLUP pedigree-based approach}
\KeywordTok{set.seed}\NormalTok{(}\DecValTok{12345}\NormalTok{)}
\NormalTok{y.trn <-}\StringTok{ }\NormalTok{DT}
\NormalTok{vv <-}\StringTok{ }\KeywordTok{sample}\NormalTok{(}\KeywordTok{rownames}\NormalTok{(DT),}\KeywordTok{round}\NormalTok{(}\KeywordTok{nrow}\NormalTok{(DT)}\OperatorTok{/}\DecValTok{5}\NormalTok{))}
\NormalTok{y.trn[vv,}\StringTok{"X1"}\NormalTok{] <-}\StringTok{ }\OtherTok{NA}
\KeywordTok{head}\NormalTok{(y.trn)}
\end{Highlighting}
\end{Shaded}

\begin{verbatim}
##              X1          X2          X3         X4   id
## 775          NA -1.72746986 -1.89028479  0.0509159  775
## 2166 -0.2527028  0.40952243  0.30938553 -1.7387588 2166
## 2167  0.3418151 -0.64862633 -0.79955921 -1.0535691 2167
## 2465         NA  0.09394919  0.57046773  0.5517574 2465
## 3881         NA -0.28248062  1.61868192 -0.1142848 3881
## 3889  2.3360969  0.62647587  0.07353311  0.7195856 3889
\end{verbatim}

\begin{Shaded}
\begin{Highlighting}[]
\NormalTok{## GBLUP}
\NormalTok{ans <-}\StringTok{ }\KeywordTok{mmer}\NormalTok{(X1}\OperatorTok{~}\DecValTok{1}\NormalTok{,}
            \DataTypeTok{random=}\OperatorTok{~}\KeywordTok{vs}\NormalTok{(id,}\DataTypeTok{Gu=}\NormalTok{K), }
            \DataTypeTok{rcov=}\OperatorTok{~}\NormalTok{units, }
            \DataTypeTok{data=}\NormalTok{y.trn) }\CommentTok{# kinship based}
\end{Highlighting}
\end{Shaded}

\begin{verbatim}
## iteration    LogLik     wall    cpu(sec)   restrained
##     1      -202.344   13:35:42      0           0
##     2      -198.717   13:35:43      1           0
##     3      -197.634   13:35:43      1           0
##     4      -197.51   13:35:44      2           0
##     5      -197.508   13:35:44      2           0
##     6      -197.508   13:35:45      3           0
\end{verbatim}

\begin{Shaded}
\begin{Highlighting}[]
\NormalTok{ans}\OperatorTok{$}\NormalTok{U}\OperatorTok{$}\StringTok{`}\DataTypeTok{u:id}\StringTok{`}\OperatorTok{$}\NormalTok{X1 <-}\StringTok{ }\KeywordTok{as.data.frame}\NormalTok{(ans}\OperatorTok{$}\NormalTok{U}\OperatorTok{$}\StringTok{`}\DataTypeTok{u:id}\StringTok{`}\OperatorTok{$}\NormalTok{X1)}
\KeywordTok{rownames}\NormalTok{(ans}\OperatorTok{$}\NormalTok{U}\OperatorTok{$}\StringTok{`}\DataTypeTok{u:id}\StringTok{`}\OperatorTok{$}\NormalTok{X1) <-}\StringTok{ }\KeywordTok{gsub}\NormalTok{(}\StringTok{"id"}\NormalTok{,}\StringTok{""}\NormalTok{,}\KeywordTok{rownames}\NormalTok{(ans}\OperatorTok{$}\NormalTok{U}\OperatorTok{$}\StringTok{`}\DataTypeTok{u:id}\StringTok{`}\OperatorTok{$}\NormalTok{X1))}
\KeywordTok{cor}\NormalTok{(ans}\OperatorTok{$}\NormalTok{U}\OperatorTok{$}\StringTok{`}\DataTypeTok{u:id}\StringTok{`}\OperatorTok{$}\NormalTok{X1[vv,],DT[vv,}\StringTok{"X1"}\NormalTok{], }\DataTypeTok{use=}\StringTok{"complete"}\NormalTok{)}
\end{Highlighting}
\end{Shaded}

\begin{verbatim}
## [1] 0.4885674
\end{verbatim}

\begin{Shaded}
\begin{Highlighting}[]
\NormalTok{## rrBLUP}
\NormalTok{ans2 <-}\StringTok{ }\KeywordTok{mmer}\NormalTok{(X1}\OperatorTok{~}\DecValTok{1}\NormalTok{,}
             \DataTypeTok{random=}\OperatorTok{~}\KeywordTok{vs}\NormalTok{(}\KeywordTok{list}\NormalTok{(GT)), }
             \DataTypeTok{rcov=}\OperatorTok{~}\NormalTok{units,}
             \DataTypeTok{data=}\NormalTok{y.trn) }\CommentTok{# kinship based}
\end{Highlighting}
\end{Shaded}

\begin{verbatim}
## iteration    LogLik     wall    cpu(sec)   restrained
##     1      -343.082   13:35:47      2           0
##     2      -243.965   13:35:47      2           0
##     3      -208.257   13:35:48      3           0
##     4      -197.982   13:35:49      4           0
##     5      -197.519   13:35:49      4           0
##     6      -197.508   13:35:50      5           0
##     7      -197.508   13:35:50      5           0
\end{verbatim}

\begin{Shaded}
\begin{Highlighting}[]
\NormalTok{u <-}\StringTok{ }\NormalTok{GT }\OperatorTok\StringTok{ }\KeywordTok{as.matrix}\NormalTok{(ans2}\OperatorTok{$}\NormalTok{U}\OperatorTok{$}\StringTok{`}\DataTypeTok{u:GT}\StringTok{`}\OperatorTok{$}\NormalTok{X1) }\CommentTok{# BLUPs for individuals}
\KeywordTok{rownames}\NormalTok{(u) <-}\StringTok{ }\KeywordTok{rownames}\NormalTok{(GT)}
\KeywordTok{cor}\NormalTok{(u[vv,],DT[vv,}\StringTok{"X1"}\NormalTok{]) }\CommentTok{# same correlation}
\end{Highlighting}
\end{Shaded}

\begin{verbatim}
## [1] 0.4885716
\end{verbatim}

\begin{Shaded}
\begin{Highlighting}[]
\CommentTok{# the same can be applied in multi-response models in GBLUP or rrBLUP}
\end{Highlighting}
\end{Shaded}

\subsection{6) Single cross prediction}\label{single-cross-prediction}

When doing prediction of single cross performance the phenotype can be
dissected in three main components, the general combining abilities
(GCA) and specific combining abilities (SCA). This can be expressed with
the same model analyzed in the diallel experiment mentioned before:

\(y = X\beta + Zu_1 + Zu_2 + Zu_S + \epsilon\)

with:

\(u_1\) \textasciitilde{} N(0, \(K_1\)\(\sigma^2_u1\))

\(u_2\) \textasciitilde{} N(0, \(K_2\)\(\sigma^2_u2\))

\(u_s\) \textasciitilde{} N(0, \(K_3\)\(\sigma^2_us\))

And we can specify the K matrices. The main difference between this
model and the full and half diallel designs is the fact that this model
will include variance covariance structures in each of the three random
effects (GCA1, GCA2 and SCA) to be able to predict the crosses that have
not ocurred yet. We will use the data published by Technow et al. (2015)
to show how to do prediction of single crosses.

\begin{Shaded}
\begin{Highlighting}[]
\KeywordTok{data}\NormalTok{(}\StringTok{"DT_technow"}\NormalTok{)}
\CommentTok{# RUN THE PREDICTION MODEL}
\NormalTok{y.trn <-}\StringTok{ }\NormalTok{DT}
\NormalTok{vv1 <-}\StringTok{ }\KeywordTok{which}\NormalTok{(}\OperatorTok{!}\KeywordTok{is.na}\NormalTok{(DT}\OperatorTok{$}\NormalTok{GY))}
\NormalTok{vv2 <-}\StringTok{ }\KeywordTok{sample}\NormalTok{(vv1, }\DecValTok{100}\NormalTok{)}
\NormalTok{y.trn[vv2,}\StringTok{"GY"}\NormalTok{] <-}\StringTok{ }\OtherTok{NA}
\NormalTok{anss2 <-}\StringTok{ }\KeywordTok{mmer}\NormalTok{(GY}\OperatorTok{~}\DecValTok{1}\NormalTok{, }
               \DataTypeTok{random=}\OperatorTok{~}\KeywordTok{vs}\NormalTok{(dent,}\DataTypeTok{Gu=}\NormalTok{Ad) }\OperatorTok{+}\StringTok{ }\KeywordTok{vs}\NormalTok{(flint,}\DataTypeTok{Gu=}\NormalTok{Af), }
               \DataTypeTok{rcov=}\OperatorTok{~}\NormalTok{units,}
               \DataTypeTok{data=}\NormalTok{y.trn) }
\end{Highlighting}
\end{Shaded}

\begin{verbatim}
## iteration    LogLik     wall    cpu(sec)   restrained
##     1      93.142   13:36:0      7           0
##     2      135.18   13:36:8      15           0
##     3      145.517   13:36:16      23           0
##     4      147.085   13:36:23      30           0
##     5      147.178   13:36:31      38           0
##     6      147.184   13:36:38      45           0
##     7      147.184   13:36:46      53           0
\end{verbatim}

\begin{Shaded}
\begin{Highlighting}[]
\KeywordTok{summary}\NormalTok{(anss2)}\OperatorTok{$}\NormalTok{varcomp}
\end{Highlighting}
\end{Shaded}

\begin{verbatim}
##                VarComp VarCompSE    Zratio Constraint
## u:dent.GY-GY  16.93639 2.6917284  6.292012   Positive
## u:flint.GY-GY 12.47174 2.3248074  5.364634   Positive
## units.GY-GY   16.75020 0.7662471 21.860045   Positive
\end{verbatim}

\begin{Shaded}
\begin{Highlighting}[]
\NormalTok{zu1 <-}\StringTok{ }\KeywordTok{model.matrix}\NormalTok{(}\OperatorTok{~}\NormalTok{dent}\OperatorTok{-}\DecValTok{1}\NormalTok{,y.trn) }\OperatorTok\StringTok{ }\NormalTok{anss2}\OperatorTok{$}\NormalTok{U}\OperatorTok{$}\StringTok{`}\DataTypeTok{u:dent}\StringTok{`}\OperatorTok{$}\NormalTok{GY}
\NormalTok{zu2 <-}\StringTok{ }\KeywordTok{model.matrix}\NormalTok{(}\OperatorTok{~}\NormalTok{flint}\OperatorTok{-}\DecValTok{1}\NormalTok{,y.trn) }\OperatorTok\StringTok{ }\NormalTok{anss2}\OperatorTok{$}\NormalTok{U}\OperatorTok{$}\StringTok{`}\DataTypeTok{u:flint}\StringTok{`}\OperatorTok{$}\NormalTok{GY}
\NormalTok{u <-}\StringTok{ }\NormalTok{zu1}\OperatorTok{+}\NormalTok{zu2}\OperatorTok{+}\NormalTok{anss2}\OperatorTok{$}\NormalTok{Beta[}\DecValTok{1}\NormalTok{,}\StringTok{"Estimate"}\NormalTok{]}
\KeywordTok{cor}\NormalTok{(u[vv2,], DT}\OperatorTok{$}\NormalTok{GY[vv2])}
\end{Highlighting}
\end{Shaded}

\begin{verbatim}
## [1] 0.8234584
\end{verbatim}

In the previous model we only used the GCA effects (GCA1 and GCA2) for
practicity, altough it's been shown that the SCA effect doesn't actually
help that much in increasing prediction accuracy and increase a lot the
computation intensity required since the variance covariance matrix for
SCA is the kronecker product of the variance covariance matrices for the
GCA effects, resulting in a 10578x10578 matrix that increases in a very
intensive manner the computation required.

A model without covariance structures would show that the SCA variance
component is insignificant compared to the GCA effects. This is why
including the third random effect doesn't increase the prediction
accuracy.

\subsection{7) Spatial modeling (using the 2-dimensional
spline)}\label{spatial-modeling-using-the-2-dimensional-spline}

We will use the CPdata to show the use of 2-dimensional splines for
accomodating spatial effects in field experiments. In early generation
variety trials the availability of seed is low, which makes the use of
unreplicated design a neccesity more than anything else. Experimental
designs such as augmented designs and partially-replicated (p-rep)
designs become every day more common this days.

In order to do a good job modeling the spatial trends happening in the
field special covariance structures have been proposed to accomodate
such spatial trends (i.e.~autoregressive residuals; ar1). Unfortunately,
some of these covariance structures make the modeling rather unstable.
More recently other research groups have proposed the use of
2-dimensional splines to overcome such issues and have a more robust
modeling of the spatial terms (Lee et al. 2013; Rodríguez-Álvarez et al.
2018).

In this example we assume an unreplicated population where row and range
information is available which allows us to fit a 2 dimensional spline
model.

\begin{Shaded}
\begin{Highlighting}[]
\KeywordTok{data}\NormalTok{(}\StringTok{"DT_cpdata"}\NormalTok{)}
\NormalTok{### mimic two fields}
\NormalTok{A <-}\StringTok{ }\KeywordTok{A.mat}\NormalTok{(GT)}
\NormalTok{mix <-}\StringTok{ }\KeywordTok{mmer}\NormalTok{(Yield}\OperatorTok{~}\DecValTok{1}\NormalTok{,}
            \DataTypeTok{random=}\OperatorTok{~}\KeywordTok{vs}\NormalTok{(id, }\DataTypeTok{Gu=}\NormalTok{A) }\OperatorTok{+}
\StringTok{              }\KeywordTok{vs}\NormalTok{(Rowf) }\OperatorTok{+}
\StringTok{              }\KeywordTok{vs}\NormalTok{(Colf) }\OperatorTok{+}
\StringTok{              }\KeywordTok{vs}\NormalTok{(}\KeywordTok{spl2D}\NormalTok{(Row,Col)),}
            \DataTypeTok{rcov=}\OperatorTok{~}\KeywordTok{vs}\NormalTok{(units),}
            \DataTypeTok{data=}\NormalTok{DT)}
\end{Highlighting}
\end{Shaded}

\begin{verbatim}
## iteration    LogLik     wall    cpu(sec)   restrained
##     1      -154.198   13:36:48      0           0
##     2      -152.064   13:36:49      1           0
##     3      -151.265   13:36:49      1           0
##     4      -151.202   13:36:49      1           0
##     5      -151.201   13:36:50      2           0
\end{verbatim}

\begin{Shaded}
\begin{Highlighting}[]
\KeywordTok{summary}\NormalTok{(mix)}
\end{Highlighting}
\end{Shaded}

\begin{verbatim}
## ============================================================
##          Multivariate Linear Mixed Model fit by REML         
## **********************  sommer 3.7  ********************** 
## ============================================================
##          logLik      AIC      BIC Method Converge
## Value -151.2011 304.4021 308.2938     NR     TRUE
## ============================================================
## Variance-Covariance components:
##                     VarComp VarCompSE Zratio Constraint
## u:id.Yield-Yield      783.4     319.3 2.4536   Positive
## u:Rowf.Yield-Yield    814.7     390.5 2.0863   Positive
## u:Colf.Yield-Yield    182.2     129.7 1.4053   Positive
## u:Row.Yield-Yield     513.6     694.7 0.7393   Positive
## u:units.Yield-Yield  2922.6     294.1 9.9368   Positive
## ============================================================
## Fixed effects:
##   Trait      Effect Estimate Std.Error t.value
## 1 Yield (Intercept)    132.1     8.791   15.03
## ============================================================
## Groups and observations:
##        Yield
## u:id     363
## u:Rowf    13
## u:Colf    36
## u:Row    168
## ============================================================
## Use the '$' sign to access results and parameters
\end{verbatim}

Notice that the job is done by the \texttt{spl2D()} function that takes
the Row and Col information to fit a spatial kernel.

\subsection{8) Multivariate genetic models and genetic
correlations}\label{multivariate-genetic-models-and-genetic-correlations}

Sometimes is important to estimate genetic variance-covariance among
traits, multi-reponse models are very useful for such task. Let see an
example with 3 traits (color, Yield, and Firmness) and a single random
effect (genotype; id) although multiple effects can be modeled as well.
We need to use a variance covariance structure for the random effect to
be able to obtain the genetic covariance among traits.

\begin{Shaded}
\begin{Highlighting}[]
\KeywordTok{data}\NormalTok{(}\StringTok{"DT_cpdata"}\NormalTok{)}
\NormalTok{A <-}\StringTok{ }\KeywordTok{A.mat}\NormalTok{(GT)}
\NormalTok{ans.m <-}\StringTok{ }\KeywordTok{mmer}\NormalTok{(}\KeywordTok{cbind}\NormalTok{(Yield,color)}\OperatorTok{~}\DecValTok{1}\NormalTok{,}
               \DataTypeTok{random=}\OperatorTok{~}\StringTok{ }\KeywordTok{vs}\NormalTok{(id, }\DataTypeTok{Gu=}\NormalTok{A)}
               \OperatorTok{+}\StringTok{ }\KeywordTok{vs}\NormalTok{(Rowf,}\DataTypeTok{Gtc=}\KeywordTok{diag}\NormalTok{(}\DecValTok{2}\NormalTok{))}
               \OperatorTok{+}\StringTok{ }\KeywordTok{vs}\NormalTok{(Colf,}\DataTypeTok{Gtc=}\KeywordTok{diag}\NormalTok{(}\DecValTok{2}\NormalTok{)),}
               \DataTypeTok{rcov=}\OperatorTok{~}\StringTok{ }\KeywordTok{vs}\NormalTok{(units),}
               \DataTypeTok{data=}\NormalTok{DT)}
\end{Highlighting}
\end{Shaded}

\begin{verbatim}
## iteration    LogLik     wall    cpu(sec)   restrained
##     1      -375.872   13:36:55      4           0
##     2      -291.932   13:36:59      8           0
##     3      -258.273   13:37:4      13           0
##     4      -253.459   13:37:8      17           0
##     5      -253.291   13:37:12      21           0
##     6      -253.278   13:37:17      26           0
##     7      -253.277   13:37:22      31           0
##     8      -253.277   13:37:26      35           0
\end{verbatim}

Now you can extract the BLUPs using the `randef' function or simple
accesing with the `\$' sign and pick `u.hat'. Also, genetic correlations
and heritabilities can be calculated easily.

\begin{Shaded}
\begin{Highlighting}[]
\KeywordTok{cov2cor}\NormalTok{(ans.m}\OperatorTok{$}\NormalTok{sigma}\OperatorTok{$}\StringTok{`}\DataTypeTok{u:id}\StringTok{`}\NormalTok{)}
\end{Highlighting}
\end{Shaded}

\begin{verbatim}
##           Yield     color
## Yield 1.0000000 0.1234441
## color 0.1234441 1.0000000
\end{verbatim}

\subsection{9) Final remarks}\label{final-remarks}

Keep in mind that sommer uses direct inversion (DI) algorithm which can
be very slow for large datasets. The package is focused in problems of
the type p \textgreater{} n (more random effect levels than
observations) and models with dense covariance structures. For example,
for experiment with dense covariance structures with low-replication
(i.e.~2000 records from 1000 individuals replicated twice with a
covariance structure of 1000x1000) sommer will be faster than MME-based
software. Also for genomic problems with large number of random effect
levels, i.e.~300 individuals (n) with 100,000 genetic markers (p). For
highly replicated trials with small covariance structures or n
\textgreater{} p (i.e.~2000 records from 200 individuals replicated 10
times with covariance structure of 200x200) asreml or other MME-based
algorithms will be much faster and we recommend you to opt for those
software.

\subsection{Literature}\label{literature}

Covarrubias-Pazaran G. 2016. Genome assisted prediction of quantitative
traits using the R package sommer. PLoS ONE 11(6):1-15.

Covarrubias-Pazaran G. 2018. Software update: Moving the R package
sommer to multivariate mixed models for genome-assisted prediction. doi:
\url{https://doi.org/10.1101/354639}

Bernardo Rex. 2010. Breeding for quantitative traits in plants. Second
edition. Stemma Press. 390 pp.

Gilmour et al. 1995. Average Information REML: An efficient algorithm
for variance parameter estimation in linear mixed models. Biometrics
51(4):1440-1450.

Henderson C.R. 1975. Best Linear Unbiased Estimation and Prediction
under a Selection Model. Biometrics vol.~31(2):423-447.

Kang et al. 2008. Efficient control of population structure in model
organism association mapping. Genetics 178:1709-1723.

Lee, D.-J., Durban, M., and Eilers, P.H.C. (2013). Efficient
two-dimensional smoothing with P-spline ANOVA mixed models and nested
bases. Computational Statistics and Data Analysis, 61, 22 - 37.

Lee et al. 2015. MTG2: An efficient algorithm for multivariate linear
mixed model analysis based on genomic information. Cold Spring Harbor.
doi: \url{http://dx.doi.org/10.1101/027201}.

Maier et al. 2015. Joint analysis of psychiatric disorders increases
accuracy of risk prediction for schizophrenia, bipolar disorder, and
major depressive disorder. Am J Hum Genet; 96(2):283-294.

Rodriguez-Alvarez, Maria Xose, et al. Correcting for spatial
heterogeneity in plant breeding experiments with P-splines. Spatial
Statistics 23 (2018): 52-71.

Searle. 1993. Applying the EM algorithm to calculating ML and REML
estimates of variance components. Paper invited for the 1993 American
Statistical Association Meeting, San Francisco.

Yu et al. 2006. A unified mixed-model method for association mapping
that accounts for multiple levels of relatedness. Genetics 38:203-208.

Abdollahi Arpanahi R, Morota G, Valente BD, Kranis A, Rosa GJM, Gianola
D. 2015. Assessment of bagging GBLUP for whole genome prediction of
broiler chicken traits. Journal of Animal Breeding and Genetics
132:218-228.

Tunnicliffe W. 1989. On the use of marginal likelihood in time series
model estimation. JRSS 51(1):15-27.


\end{document}
